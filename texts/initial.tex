\begin{titlepage}
    \bfseries{
        \begin{center}
            \LARGE{STŘEDOŠKOLSKÁ ODBORNÁ ČINNOST}

            \vspace{14pt}
            \large{
                Obor č. 18: Informatika
            }

            \vspace{0.4 \textheight}

            \LARGE{
			Systém pro vkládání a zpracování ankety/zpětné vazby
            }

            \vspace{0.3\textheight}
        \end{center}
        
        \noindent\Large{Jan Hubený}

        \noindent\Large{Pardubický kraj \hspace{\stretch{1}}  Pardubice, 2022}
        
            
    }
\end{titlepage}

\cleardoublepage

%% Úvodní stránka s informacemi
{\bfseries
    \begin{center}
        \LARGE{STŘEDOŠKOLSKÁ ODBORNÁ ČINNOST}

        \vspace{14pt}
        {\large
            Obor č. 18: Informatika
        }

        \vspace{0.3 \textheight}

        \LARGE{
        Systém pro vkládání a zpracování ankety/zpětné vazby
        }

        %%\LARGE{HansForms}

        \vspace{0.24\textheight}
    \end{center}  
}
{\Large
    \noindent\textbf{Autor:} Jan Hubený\\
    \textbf{Škola:} DELTA - Střední škola informatiky a ekonomie, s.r.o.\\
    \textbf{Kraj:} Pardubický kraj\\
    \textbf{Konzultant:} RNDr. Jan Koupil, Ph.D.\\
}

\noindent Pardubice, 2022

\cleardoublepage

\noindent{\Large{\bfseries{Prohlášení}}}

\noindent Prohlašuji, že jsem svou práci SOČ vypracoval/a samostatně a použil/a jsem pouze prameny a literaturu uvedené v~seznamu bibliografických záznamů.

\noindent Prohlašuji, že tištěná verze a elektronická verze soutěžní práce SOČ jsou shodné.

\noindent Nemám závažný důvod proti zpřístupňování této práce v~souladu se zákonem č. 121/2000 Sb., o~právu autorském, o~právech souvisejících s~právem autorským a o~změně některých zákonů (autorský zákon) ve znění pozdějších předpisů.  

\vspace{24 pt}

\noindent V~Pardubicích dne 15. Března 2022 \dotfill{}\hspace{\stretch{0.4}} 

\hspace{8cm} Jan Hubený

\cleardoublepage

\vspace*{0.8\textheight}
\noindent{\Large{\bfseries{Poděkování}}}

\noindent
Prvně bych chtěl poděkovat mému vedoucímu práce RNDr. Janu Koupilovi, Ph.D. za odborné vedení projektu a věcné připomínky. Dále děkuji Bc. Tomáši Vanišovi za pomoc při prvotních krocích při tvorbě aplikace a za pomoc při výběru technologií pro tento projekt. V~neposlední řadě také děkuji své rodině, přátelům a spolužákům za psychickou podporu, za účast na testování aplikace a za nápady pro zlepšení aplikace.

\cleardoublepage

\noindent{\Large{\bfseries{Anotace}}}

\noindent 
Práce dokumentuje vývoj a používání webové aplikace pro zpracování ankety nebo zpětné vazby. Systém je založen na backendovém frameworku Laravel, frontendovém frameworku VueJS a na databázovém systému PostgreSQL. Uživatele v aplikaci mohou jednak formuláře vyplňovat, ale i spravovat. Kromě základní funkcionality jako je vykreslení formuláře a následná možnost odpovědi aplikace obsahuje např. omezení přístupu datem spuštění a ukončení, privátní přístup pozvaným uživatelům či export dat pro účely korelační analýzy.

\vspace{18pt}

\noindent{\Large{\bfseries{Klíčová slova}}}

\noindent Webová aplikace; Zpracování ankety; Laravel, VueJS, PostgreSQL

\vspace{18pt}

\noindent{\Large{\bfseries{Annotation}}}

\noindent
The thesis documents the development and use of a web application for processing surveys or feedback. The system is based on the backend framework Laravel, the frontend framework VueJS and the database system PostgreSQL. Users can both fill out and manage forms in the application. In addition to basic functionality such as form rendering and subsequent response option, the application includes for example access restriction by start and end date, private access for invited users or data export for correlation analysis.

\vspace{18pt}

\noindent{\Large{\bfseries{Keywords}}}

\noindent Web application; Processing surveys; Laravel, VueJS, PostgreSQL

\cleardoublepage

\tableofcontents

\cleardoublepage