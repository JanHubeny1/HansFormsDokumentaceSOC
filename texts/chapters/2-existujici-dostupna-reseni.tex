\chapter{Existující dostupná řešení}

Webových systému pro zpracování ankety je na trhu opravdu nepřeberné množství. Představme si proto alespoň některé z~nich, jelikož byly inspirací pro mou vlastní implementaci.

\section{Google Forms}

Google Forms (česky Google Formuláře) je webová aplikace určená k~tvorbě online průzkumů a kvízů. Je součástí balíčku webových aplikací od Googlu, ve kterých nalezneme např. Google Sheets či Google Docs. Při tvorbě těchto formulářů máme možnost pracovat s~různými předpřipravenými šablonami nebo s~předdefinovanými typy otázek. Do formulářů můžeme komponovat i různá multimédia jako např. videa či obrázky a samozřejmě zde nechybí i vizualizace výsledků. Jediné, co potřebujeme k~tvorbě formulářů na této platformě, je vytvoření a přihlášení do Google účtu. \cite{GoogleForms1}\cite{GoogleForms2}

\section{Microsoft Forms}
Microsoft Forms (česky Microsoft Formuláře) je služba umožňující tvorbu a sdílení dotazníků a kvízů určený jak pro soukromé účely, tak pro firemní sféru. Je součástí balíčku aplikací Microsoft Office, kam patří např. Microsoft Word nebo Microsoft Excel. I~zde máme při tvorbě ankety možnost vybrat z~předdefinovaných typů otázek společně s~např. možností nahrání souboru či zobrazení výsledků. K~tvorbě ankety se stačí jen přihlásit do svého Microsoft účtu. \cite{MSForms1}\cite{MSForms2}

\section{Survio}
Survio je anketní nástroj pro měření zákaznické spokojenosti, marketingový průzkum či jiné účely původem z~Česka. Kromě samotného tvoření formuláře nabízí i funkce jak např. export dotazníku do PDF, souhrnnou statistiku ve grafech a tabulkách či zpracování výsledků v~reálném čase. Tvoření ankety se základními funkcemi je zdarma, služba ale nabízí i možnost prémiových funkcí. K~tvorbě formuláře je znovu třeba se přihlásit. \cite{Survio}

