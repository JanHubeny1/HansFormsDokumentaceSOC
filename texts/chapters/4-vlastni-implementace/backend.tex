\section{Backend}
	V rámci backendové nebo-li serverové části projektu je prvně nutné zmínit, že funguje na principu API (???citace???) - tzn. že na ní přicházejí požadavky od klienta a na ně náležitě odpovídá. Nevrací ani nijak nespravuje vizualizaci dat - jen předává data na frontendovou část, o které je psáno dále.
	
	Zde jsou zobecněné základní úkony, které v mé implementaci právě backend vykonává:
	\begin{itemize}
		\item Zpracování a vyřizování požadavků z webové frontendové části
		\item Administraci databáze - tzn. vytváření, úpravu, mazání a čtení jednotlivých objektů a migrace databáze (tj. automatizovaná deklarace struktury databázového objektu ???citace???)
		\item Rozesílání emailů (např. pozvánky na neveřejný formulář)
		\item Exportování dat do Excel tabulek
		\item Autentifikaci uživatele
	\end{itemize}
	 
	\subsection{Obecná struktura Laravel projektu}
	%% obrázek stromové struktury složky

	\subsection{Modely}
			
	\subsection{Kontrolery}
	
	\subsection{Exporty}
	
	\subsection{Maily}
	
	\subsection{Migrace databáze}
		\subsubsection{Factories}
	
	\subsection{Pohledy}
	
	\subsection{Routes}
	
	\subsection{Průběh jednotlivých činností}
		\subsubsection{Ukládání formuláře}
		\subsubsection{Mazání formuláře}
		%% atd.


